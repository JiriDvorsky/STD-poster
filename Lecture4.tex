\block{Lekce IV -- Tabulky}
{
	Čitelné, profesionálně vysázené tabulky jsou nezbytné pro správnou prezentaci jakýchkoliv číselných údajů, například pro představení výsledků provedených experimentů. Lekce zahrnuje tato témata:
	\begin{itemize}
		\item Základní sazba tabulek, prostředí \GeneralName{tabular}
		\item Definice sloupců
%			\begin{itemize}
%				\item Sloupce \GeneralName{l}, \GeneralName{r} a \GeneralName{c}
%				\item Sloupce \GeneralName{p}, \GeneralName{b} a \GeneralName{m}
%				\item Sloupce zarovnané na desetinou čárku
%			\end{itemize}
		\item Vstup dat
		\item Víceřádkové a vícesloupcové buňky
		\item Ohraničení tabulek
			\begin{itemize}
				\item Obecná pravidla
				\item Ohraničení ve standardním \hologo{LaTeX}u
				\item Balík maker \GeneralName{booktabs}
			\end{itemize}
		\item Tabulky jako plovoucí objekty
			\begin{itemize}
				\item Prostředí \GeneralName{table}
				\item Popisky a křížové odkazy
%				\item Umisťování tabulek v textu
			\end{itemize}
		\item Další možnosti sazby tabulek
			\begin{itemize}
%				\item Sazba tabulek naležato
				\item Vícestránkové tabulky
				\item Tabulky a barvy
			\end{itemize}
	\end{itemize}
}
\endinput
